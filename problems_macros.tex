\documentclass[12pt]{article}
\usepackage[T2A]{fontenc}
\usepackage[utf8]{inputenc}        % Кодировка входного документа;
                                    % при необходимости, вместо cp1251
                                    % можно указать cp866 (Alt-кодировка
                                    % DOS) или koi8-r.

\usepackage[english,russian]{babel} % Включение русификации, русских и
                                    % английских стилей и переносов





%%%%%%%%%%%%%%%%%%%%%%%%%%%%%%% Problems macros  %%%%%%%%%%%%%%%%%%%%%%%%%%%%%%%
\usepackage{amsmath,amsfonts,amsthm,amssymb,amsbsy,amstext,amscd,amsxtra}
\usepackage{verbatim}
\usepackage{tikz}
\usepackage{graphicx}
\usepackage[colorlinks,urlcolor=blue]{hyperref}
\usepackage[stable]{footmisc}


%%%%%%%%%%%%%%%%%%%%%%%% Enumerations %%%%%%%%%%%%%%%%%%%%%%%%

\newcommand{\Rnum}[1]{\expandafter{\romannumeral #1\relax}}
\newcommand{\RNum}[1]{\uppercase\expandafter{\romannumeral #1\relax}}

%%%%%%%%%%%%%%%%%%%%% EOF Enumerations %%%%%%%%%%%%%%%%%%%%%

\usepackage{xparse}
\usepackage{ifthen}
\usepackage{bm} %%% bf in math mode
\usepackage{color}

\definecolor{Gray555}{HTML}{555555}
\definecolor{Gray444}{HTML}{444444}
\definecolor{Gray333}{HTML}{333333}


\newcounter{problem}
\newcounter{uproblem}
\newcounter{kvproblem}
\newcounter{subproblem}
\newcounter{subrproblem}
\renewcommand{\thesubrproblem}{\asbuk{subrproblem}}
\newcounter{prvar}
\newcounter{solvar}

\def\slv{\noindent\stepcounter{solvar}{\bf\RNum{\thesolvar}}. \setcounter{prvar}{0}\setcounter{subproblem}{0}\setcounter{subrproblem}{0}}

\def\solvend{\qed\bigskip}


\usepackage{pgfkeys}


\newcommand\prtemplate[2][]{%
 \pgfkeys{/prtemplate,name,counter,label,display,type,footnote,#1,print}\ifthenelse{\equal{#2}{}}{}{#2 }% 
}
\pgfkeys{
 /prtemplate/.is family, /prtemplate,
 name/.default = {},
 name/.store in = \prname,
 label/.default = {},
 label/.store in = \prlabel,
 footnote/.default = {},
 footnote/.store in = \prfootnote, 
 counter/.default = {problem},
 counter/.store in = \prcounter,
 display/.store in=\prdisplay,
 type/.default={},
 type/.store in=\prtype,
 print/.code =  {\ifthenelse{\equal{\prdisplay}{inline}}{}{\medskip\noindent}\ifthenelse{\equal{\prlabel}{}}{\stepcounter{\prcounter}}{\refstepcounter{\prcounter}\label{\prlabel}}{\bf \ifthenelse{\equal{\prname}{}}{}{\prname~}\prwrite[\prtype]\ifthenelse{\equal{\prfootnote}{}}{}{\footnote{\prfootnote}}.\;}\setcounter{subproblem}{0}\setcounter{subrproblem}{0}}}



\newcommand{\prwrite}[1][]{\ifthenelse{\equal{#1}{star}}{$\mathbf{\arabic{\prcounter}}^{*}$}{\ifthenelse{\equal{#1}{o}}{$\mathbf{\arabic{\prcounter}}^{\circ}$}{\textbf{\arabic{\prcounter}}}}}


\newcommand\prsubrtemplate[2][]{
 \pgfkeys{/prsubrtemplate,label,display,type,#1,print}\ifthenelse{\equal{#2}{}}{}{#2 }% 
}
\pgfkeys{
 /prsubrtemplate/.is family, /prsubrtemplate,
 label/.default = {},
 label/.store in = \prlabel,
 display/.default={},
 display/.store in=\prdisplay,
 type/.default={},
 type/.store in=\prstype,
 print/.code =  {\ifthenelse{\equal{\prdisplay}{inline}}{}{\medskip\noindent}\ifthenelse{\equal{\prlabel}{}}{\stepcounter{subrproblem}}{\refstepcounter{subrproblem}\label{\prlabel}}{\bf \prsubrwrite[\prstype]\;}}
}
\newcommand{\prsubrwrite}[1][]{\ifthenelse{\equal{#1}{star}}{{\bf \asbuk{subrproblem})$^{*}$\!}}
	{\ifthenelse{\equal{#1}{o}}{{\bf \asbuk{subrproblem})$^{\circ}$\!}}
		{\textbf{\asbuk{subrproblem})}}}}

\newcommand\prsubr[1][]{\prsubrtemplate[#1]{}}
\newcommand\prsubru[1][]{\prsubrtemplate[#1,type=o]{}}
\newcommand\prsubrinline[1][]{\prsubrtemplate[#1,display=inline]{}}
\newcommand\prsubruinline[1][]{\prsubrtemplate[#1,type=o,display=inline]{}}
\newcommand\prsubrstar[1][]{\prsubrtemplate[#1,type=star]{}}
\newcommand\prsubrstarinline[1][]{\prsubrtemplate[#1,type=star,display=inline]{}}


\newcommand\prsubtemplate[2][]{%
 \pgfkeys{/prsubtemplate,label,display,type,#1,print}\ifthenelse{\equal{#2}{}}{}{#2 }% 
}
\pgfkeys{
 /prsubtemplate/.is family, /prsubtemplate,
 label/.default = {},
 label/.store in = \prlabel,
 display/.default={},
 display/.store in=\prdisplay,
 type/.default={},
 type/.store in=\prstype,
 print/.code =  {\ifthenelse{\equal{\prdisplay}{inline}}{}{\medskip\noindent}\ifthenelse{\equal{\prlabel}{}}{\stepcounter{subproblem}}{\refstepcounter{subproblem}\label{\prlabel}}\setcounter{subrproblem}{0}{\sf \prsubwrite[\prstype]\;}}
}
\newcommand{\prsubwrite}[1][]{\ifthenelse{\equal{#1}{star}}{{ $\mathsf{\thesubproblem}^*.$}}
	{\ifthenelse{\equal{#1}{o}}{{$\mathsf{\thesubproblem}^{\circ}.$}}
		{\thesubproblem.}}}

\newcommand\prsub[1][]{\prsubtemplate[#1]{}}
\newcommand\prsubu[1][]{\prsubtemplate[#1,type=o]{}}
\newcommand\prsubinline[1][]{\prsubtemplate[#1,display=inline]{}}
\newcommand\prsubuinline[1][]{\prsubtemplate[#1,type=o,display=inline]{}}
\newcommand\prsubstar[1][]{\prsubtemplate[#1,type=star]{}}
\newcommand\prsubstarinline[1][]{\prsubtemplate[#1,type=star,display=inline]{}}


\newenvironment{quiz}[1][]{\prtemplate[name={Контрольный вопрос}, counter={kvproblem}, #1]{}}{\medskip}
\newenvironment{upr}[1][]{\prtemplate[name={Упражнение}, counter={uproblem}, #1]{}}{\medskip}
\newenvironment{problem}[1][]{\prtemplate[name={Задача}, counter={problem}, #1]{}}{\medskip}
\newcommand\pr[1][]{\prtemplate[#1]{}}







\def\prinline{\noindent\stepcounter{problem}{\bf \theproblem .\;}\setcounter{subproblem}{0}\setcounter{subrproblem}{0}}

\def\pru{\pr[type=o]}
\def\pruinline{\pr[type=o,display=inline]}
\def\prstar{\pr[type=star]}
\def\prpfrom[#1]{\medskip\noindent\stepcounter{problem}{\bf Задача \theproblem~(№#1 из задания).  }\setcounter{subproblem}{0}\setcounter{subrproblem}{0} }
\def\prp{\medskip\noindent\stepcounter{problem}{\bf Задача \theproblem .}\setcounter{subproblem}{0}\setcounter{subrproblem}{0}\;}

\def\prpvar{\medskip\noindent\stepcounter{problem}\setcounter{prvar}{1}\setcounter{solvar}{1}{\bf Задача \theproblem \;$\langle${\rm\Rnum{\theprvar}}$\rangle$.}\setcounter{subproblem}{0}\setcounter{subrproblem}{0}\;}
\def\prpv{\medskip\noindent\stepcounter{prvar}{\bf Задача \theproblem \,$\bm\langle$\bracketspace{{\rm\Rnum{\theprvar}}}$\bm\rangle$. }\setcounter{subproblem}{0}\setcounter{subrproblem}{0}}
\def\prv{\medskip\noindent\stepcounter{prvar}{\bf \theproblem\,$\bm\langle$\bracketspace{{\rm\Rnum{\theprvar}}}$\bm\rangle$}.\setcounter{subproblem}{0}\setcounter{subrproblem}{0} }


\def\prpstar{\medskip\noindent\stepcounter{problem}{\bf Задача $\bf\theproblem^*$\negthickspace.  }\setcounter{subproblem}{0}\setcounter{subrproblem}{0} }
\def\prdag{\medskip\noindent\stepcounter{problem}{\bf Задача $\theproblem^{^\dagger}$\negthickspace\,.  }\setcounter{subproblem}{0}\setcounter{subrproblem}{0} }



\newcommand{\bracketspace}[1]{\phantom{(}\!\!{#1}\!\!\phantom{)}}

\DeclareDocumentCommand{\Prpvar}{ O{null} O{} }{
	\medskip\noindent\stepcounter{problem}\setcounter{prvar}{1}\setcounter{solvar}{1}{\bf Задача \theproblem
	\,{\sf $\bm\langle$\bracketspace{{\rm\Rnum{\theprvar}}}$\bm\rangle$}
	~\!\!\! \ifthenelse{\equal{#1}{null}}{\!}{{\sf(\bracketspace{#1})}}}.}

\DeclareDocumentCommand{\Prp}{ O{null} O{null} }{\setcounter{subproblem}{0}\setcounter{subrproblem}{0}
	\medskip\noindent\stepcounter{problem}\setcounter{prvar}{0}\setcounter{solvar}{0}{\bf Задача \theproblem
	~\!\!\! \ifthenelse{\equal{#1}{null}}{\!}{{\sf(\bracketspace{#1})}}
	 \ifthenelse{\equal{#2}{null}}{\!\!}{{\sf [\color{Gray444}\,\bracketspace{#2}\,]}}}.}

\DeclareDocumentCommand{\Pr}{ O{null} O{null} }{\setcounter{subproblem}{0}\setcounter{subrproblem}{0}
	\medskip\noindent\stepcounter{problem}\setcounter{prvar}{0}\setcounter{solvar}{0}{\bf\theproblem
	~\!\!\! \ifthenelse{\equal{#1}{null}}{\!}{{\sf(\bracketspace{#1})}}
	 \ifthenelse{\equal{#2}{null}}{\!\!}{{\sf [\color{Gray444}\,\bracketspace{#2}\,]}}}.}


\DeclareDocumentCommand{\Prstar}{ O{null} O{null} }{\setcounter{subproblem}{0}\setcounter{subrproblem}{0}
		\medskip\noindent\stepcounter{problem}\setcounter{prvar}{0}\setcounter{solvar}{0}{\bf$\mathbf{\theproblem^*}$
		~\!\!\! \ifthenelse{\equal{#1}{null}}{\!}{{\sf(\bracketspace{#1})}}
		 \ifthenelse{\equal{#2}{null}}{\!\!}{{\sf [\color{Gray444}\,\bracketspace{#2}\,]}}}.}


\DeclareDocumentCommand{\Prps}{ O{null} O{null} }{\setcounter{subproblem}{0}\setcounter{subrproblem}{0}
	\medskip\noindent\stepcounter{problem}\setcounter{prvar}{0}\setcounter{solvar}{0}{\bf Задача $\bm\theproblem^* $
	~\!\!\! \ifthenelse{\equal{#1}{null}}{\!}{{\sf(\bracketspace{#1})}}
	 \ifthenelse{\equal{#2}{null}}{\!\!}{{\sf [\color{Gray444}\,\bracketspace{#2}\,]}}}.
}

\DeclareDocumentCommand{\Prpd}{ O{null} O{null} }{\setcounter{subproblem}{0}\setcounter{subrproblem}{0}
	\medskip\noindent\stepcounter{problem}\setcounter{prvar}{0}\setcounter{solvar}{0}{\bf Задача $\bm\theproblem^\dagger$
	~\!\!\! \ifthenelse{\equal{#1}{null}}{\!}{{\sf(\bracketspace{#1})}}
	 \ifthenelse{\equal{#2}{null}}{\!\!}{{\sf [\color{Gray444}\,\bracketspace{#2}\,]}}}.
}


\def\prend{
	\medskip
%	\bigskip
}


\def\prskip{
	\smallskip
}


%%%%%%%%%%%%%%%%%%%%%%%%%%%%%%% EOF Problems macros  %%%%%%%%%%%%%%%%%%%%%%%%%%%%%%%

\newcommand{\comments}[2][Комментарий]{
\medskip
	\noindent{\bf #1: }{\sl #2}
%\medskip	
}


%%%%%%%%%%%%%%%%%%%%%%%%%%%%%%% Page style  %%%%%%%%%%%%%%%%%%%%%%%%%%%%%%%%прочее


                                     %\fontseries{m}\selectfont
\title{Макросы для работы с задачами}
\author{Александр Рубцов \\ \texttt{\small\href{mailto:alex@rubtsov.su}{alex@rubtsov.su}}}

\date{}

\usepackage{indentfirst}

\usepackage{minitoc}
\usepackage{tocvsec2}




%%%%%%%%%%%%%%%%%%%%%%%%%%%%%%%%%%%%%%%%%%%%%%%%%%%% Отображение решений %%%%%%%%%%%%%%%%%%%%%%%%%%%%%%%%%%%%%%%%%%%%%%%%%%%%
\newif\ifsolshow
%\solshowfalse
\solshowtrue

\usepackage{comment}
\ifsolshow
	\newenvironment{sol}
			{	            
				\bigskip
				\setcounter{prvar}{0}\setcounter{subproblem}{0}\noindent\textbf{Решение.}\;}
			{		
			}    
		 	\newenvironment{ukaz}
				{	
					\bigskip
					\setcounter{prvar}{0}\setcounter{subproblem}{0}\noindent\textbf{Указания.}\;}
				{		
				}
		 \newenvironment{crit}
				{	        
					\bigskip
					\setcounter{prvar}{0}\setcounter{subproblem}{0}\noindent\textbf{Критерии.}\;}
				{		
				}
	 	\newenvironment{discussion}
			{	        
				\bigskip
				\setcounter{prvar}{0}\setcounter{subproblem}{0}\noindent\textbf{Рабочее обсуждение.}\;}
			{		
			}	
	\else
		\excludecomment{sol} 
		\excludecomment{ukaz}
		\excludecomment{crit}
		\excludecomment{discussion}
	\fi


\usepackage{fancyvrb}


\begin{document}
	%\lstset{language=TeX}
	\renewcommand{\partname}{}
	\doparttoc


\maketitle

В этом файле приведены макросы для задач, который я использую для составления листков, контрольных, написания конспектов и книжек. Макросы находятся в стадии development уже довольно давно; аккуратно всё причесать и доработать весь функционал руки не доходят уже много лет, а коллеги время от времени спрашивают как сделать то или иное (например, ссылки на задачи), поэтому я решил выложить макросы в том виде, в котором они есть. В этом виде они были использованы для написания книжки и постоянно используются для составления листков, контрольных работ и подборок задач.


\tableofcontents

\newpage


%\setcounter{section}{-1}


\section{Основные макросы}

\subsection{Задачи и их виды}

В листках и контрольных стандартно используется нумерация задач жирным макросом

\pr Найти сумму $1+2+\ldots+n$.

\begin{Verbatim}[frame=single]
\pr Найти сумму $1+2+\ldots+n$.
\end{Verbatim}

В конспектах лекций и книжках к задаче добавляется префикс

\problem Найти сумму $1+2+\ldots+n$.

\begin{Verbatim}[frame=single]
\problem Найти сумму $1+2+\ldots+n$.
\end{Verbatim}

Можно создавать задачи разных видов (описывать их разными префиксами)

\upr Найти сумму $1+2+\ldots+n$.

\begin{Verbatim}[frame=single]
\upr Найти сумму $1+2+\ldots+n$.
\end{Verbatim}

Для создания окружения используйте шаблон

\begin{Verbatim}[frame=single]
\newcounter{uproblem}
\newenvironment{upr}[1][]{
	\prtemplate[name={Упражнение},
	 counter={uproblem}, #1]{}
}{\medskip}	% отступ после окончания окружения
\end{Verbatim}

После создания шаблона можно использовать задачи как через команду, так и через окружение:

\begin{Verbatim}[frame=single]
\begin{upr}
	Найти сумму $1+2+\ldots+n$.
\end{upr}
\upr Найти сумму $1+2+\ldots+n$.	
\prend % отступ после задачи
\end{Verbatim}


\begin{upr}
	Найти сумму $1+2+\ldots+n$.
\end{upr}

\upr Найти сумму $1+2+\ldots+n$.
\prend % отступ после задачи

Как видно из примеров, задачи и упражнения используют разные счётчики, поэтому нумеруются независимо.

\subsection{Подзадачи}

В макросах используются два вида подзадач. Первый~"--- для задач с несколькими различными подзадачами.

\problem Пусть $f(x) = x^2$, $g(x) = x^3$

\prsub Является ли $f(x)$ чётной функцией?

\prsub Вычислите $f'(x)$ и $g'(x)$.

\prend

\begin{Verbatim}[frame=single]
\problem Пусть $f(x) = x^2$, $g(x) = x^3$

\prsub Является ли $f(x)$ чётной функцией?

\prsub Вычислите $f'(x)$ и $g'(x)$.

\prend	
\end{Verbatim}


Второй~"--- для задач с перечислением однотипных подзадач.


\problem Для каждой функции $f(x)$ вычислите $f'(x)$:

\prsubr $x$;\quad \prsubrinline $x^2$; \quad \prsubrinline $x^3$;

\prend

\begin{Verbatim}[frame=single]
\problem Для каждой функции $f(x)$ вычислите $f'(x)$:

\prsubr $x$;\quad \prsubrinline $x^2$; 
\quad \prsubrinline $x^3$;

\prend	
\end{Verbatim}


Команда \texttt{prsubr} делает отступ от предыдущей строки, а \texttt{prsubrinline} нет. Поэтому при перечислении подзадач в одной строке, нужно сначала использовать \texttt{prsubr}, а затем \texttt{prsubrinline}.

\subsection{Пометки}

Часто бывает нужно отметить задачи повышенной сложности звёздочкой или ввести други типовые пометки. В макросах уже реализованы следующие пометки

\problem[type=star] Задача повышенной сложности

\prsub[type=star] Пункт повышенной сложности

\prsubr[type=star] подпункт повышенной сложности.

\prsub[type=o] Пункт, который планируется разобрать

\prsubr[type=o] подпункт; \prsubr[type=circ, display=inline] следующий подпункт.

\prend

\begin{Verbatim}[frame=single]
\problem[type=star] Задача повышенной сложности

\prsub[type=star] Пункт повышенной сложности

\prsubr[type=star] подпункт повышенной сложности.

\prsub[type=o] Пункт, который планируется разобрать

\prsubr[type=o] подпункт; 
\prsubr[type=circ, display=inline] следующий подпункт.

\prend
\end{Verbatim}


\pru Для типичных задач и подзадач введены сокращения. 

\prsubstar Подзадача

\prsubru подподзадача;

\prsubrstar подподзадача.


\prsubu Подзадача \prsubruinline подподзадача.

\prend

\begin{Verbatim}[frame=single]
\pru Для типичных задач и подзадач введены сокращения. 

\prsubstar Подзадача

\prsubru подподзадача;

\prsubrstar подподзадача.


\prsubu Подзадача \prsubruinline подподзадача.

\prend	
\end{Verbatim}


Обратите внимание, что если вы используете сокращения, например, \texttt{prsubu}, то в них нельзя передавать другие вспомогательные аргументы. Так, \texttt{prsubu[display=inline]} не сработает.

\subsection{Ссылки}

Нумерация задач может меняться из-за добавления или удаления других задач. На задачи можно ссылаться, используя метки. 

\problem[label=pr:labeled] Ответьте на следующие вопросы о функциях.

\prsub[label=prsub:labeled1] Какие из следующих функций чётные:

\prsubr $x$;\quad \prsubrinline $x^2$; \quad \prsubr[display=inline, type=o, label=prsubr:labeled1] $x^3$?

\prsub[label=prsub:labeled2] Вычислите производные функций

\prsubr[label=prsubr:labeled2] $\sin x$;\quad \prsubrinline $\sin^2 x$; \quad \prsubrinline $\cos 2x$.

\prend

\textbf{Указания:} в пункте \ref{pr:labeled}.\ref{prsub:labeled1}.\ref{prsubr:labeled1} функция не является чётной; в пункте \ref{pr:labeled}.\ref{prsub:labeled2}.\ref{prsubr:labeled2} ответ $\cos x$.

\begin{Verbatim}[frame=single]
\problem[label=pr:labeled] Ответьте на следующие вопросы 
о функциях.

\prsub[label=prsub:labeled1] Какие из следующих функций чётные:

\prsubr $x$;\quad \prsubrinline $x^2$; \quad 
\prsubr[display=inline, type=o, label=prsubr:labeled1] $x^3$?

\prsub[label=prsub:labeled2] Вычислите производные функций

\prsubr[label=prsubr:labeled2] $\sin x$;
\quad \prsubrinline $\sin^2 x$; 
\quad \prsubrinline $\cos 2x$.

\prend

\textbf{Указания:} в пункте 
\ref{pr:labeled}.\ref{prsub:labeled1}.\ref{prsubr:labeled1}
функция не является чётной; в пункте 
\ref{pr:labeled}.\ref{prsub:labeled2}.\ref{prsubr:labeled2} 
ответ $\cos x$.
\end{Verbatim}


\subsection{Основные опции макросов}

Повторим кратко вышеперечисленное. Стандартные задачи вызываются командой или окружением \texttt{problem}, задачи без номеров~"--- командой \texttt{pr}. Можно ввести задачи отдельных типов, таких как \texttt{upr} и \texttt{quiz} уже включённых в макросы, для которых всё работает так же как и для стандартных задач. Для подзадач используйте команды \texttt{prsub} или \texttt{prsubr}, в зависимости от типа подзадач. Во всех этих командах доступны следующие опции:

\begin{itemize}
	\item[\texttt{type}] значение \texttt{star} для пометки~$*$ и \texttt{o} для пометки~$\circ$ 
	\item[\texttt{label}] значение задаёт стандартную метку (соответствующую счётчику задачи или подзадачи), на которую потом можно сослаться через \texttt{ref}
	\item[\texttt{display}] поставьте значение \texttt{inline}, если требуется перечислить несколько пунктов \texttt{prsubr} в одной строке (только для \texttt{prsubr}). 
\end{itemize}

При введении нового типа задач выбирается счётчик. Стандартный счётчик называется \texttt{problem}. Выставить его значение (часто полезно обнулить счётчик в новом разделе) можно с помощью команды

\begin{Verbatim}[frame=single]
\setcounter{problem}{2}
\problem Это задача №3.
\end{Verbatim}

\setcounter{problem}{2}
\problem Это задача №3.

\section{Дополнительные макросы}

Макросы в этом разделе полезны при работе над контрольными. Они сделаны отдельно от основных макросов по историческим причинам: основные макросы были сделаны для работы над книжкой, а на добавление функционала для контрольных не хватило сил. Фактически в дополнительных макросах нельзя только делать ссылки на задачи; при этом макросы для подазадач используются из основных макросов.


\prp Это задача с префиксом, но без функционала макроса. Команда осталась по историческим причинам

\begin{Verbatim}[frame=single]
\prp Это задача с префиксом, но без функционала макроса. 
Команда осталась по историческим причинам
\end{Verbatim}

\subsection{Макросы для контрольных}

\Pr[4] В контрольных мы часто используем задачи с разной стоимостью в баллах.
Стоимость указана в скобках.


\begin{Verbatim}[frame=single]
\Pr[4] В контрольных мы часто используем задачи с разной 
стоимостью в баллах. Стоимость указана в скобках.	
\end{Verbatim}

При составлении контрольной или файла с подборкой задач для курса удобно указывать метаданные

\Prp[4-6][А.Р.] Эту задачу добавил преподаватель А.Р. Стоимость он оценивает от 4 до 6 баллов. Как видно у макроса есть некоторые проблемы с пробелами; решить их не получилось.

\Pr[4-6][А.Р.] Для сокращённого макроса тоже работает. Второй аргумент с мета-данными можно опускать.

\begin{Verbatim}[frame=single]
\Prp[4-6][А.Р.] Эту задачу добавил преподаватель А.Р. 
Стоимость он оценивает от 4 до 6 баллов. 
Как видно у макроса есть некоторые проблемы с пробелами;
 решить их не получилось.

\Pr[4-6][А.Р.] Для сокращённого макроса тоже работает. 
Второй аргумент с мета-данными можно опускать.
\end{Verbatim}

Также добавлен функционал для работы с задачами с вариантами (используется при составлении контрольных).

\Prpvar[4-6] Это первый вариант задачи.

\prpv Это второй вариант задачи

\Prp[3] Продифференцируйте функцию

\prv $\sin x$

\prv $\cos x$

\prv $\tg x$


\begin{Verbatim}[frame=single]
\Prpvar[4-6] Это первый вариант задачи.

\prpv Это второй вариант задачи

\Prp[3] Продифференцируйте функцию

\prv $\sin x$

\prv $\cos x$

\prv $\tg x$
\end{Verbatim}

\subsection{Обсуждение задач и решения}

В этом разделе приведены несколько дополнительных макросов, относящихся к обсуждению задач и записи решений. При совместной работе над задачами удобно отдельно выделять обсуждения

\Prp[3] Продифференцируйте функцию

\prv $\sin x$

\prv $\cos x$

\prv $\tg x$

\comments[А.Р.]{Кто добавил эту задачу? Не забывайте про метаданные! Варианты неравноценны по сложности}


\begin{Verbatim}[frame=single]
\comments[А.Р.]{Кто добавил эту задачу? 
Не забывайте про метаданные! 
Варианты неравноценны по сложности}	
\end{Verbatim}

В макросе для записи решений происходит обнуление всех счётчиков, относящихся к задаче.

\begin{sol}
	Приведём только ответы: 
	
	\prv $\cos x$; 
	
	\prv $- \sin x$; 
	
	\prv $\displaystyle\frac{1}{\cos^2 x}$.
\end{sol}

\begin{Verbatim}[frame=single]
\begin{sol}
	Приведём только ответы: 
	
	\prv $\cos x$; 
	
	\prv $- \sin x$; 
	
	\prv $\displaystyle\frac{1}{\cos^2 x}$.
\end{sol}
\end{Verbatim}



Решения можно убрать из файла, путём переключения макросов. По-умолчанию отображение решений включено:

\begin{Verbatim}[frame=single]
\newif\ifsolshow
%\solshowfalse
\solshowtrue
\end{Verbatim}

Замена параметра исключает решения из файла:
\begin{Verbatim}[frame=single]
\newif\ifsolshow
\solshowfalse
%\solshowtrue	
\end{Verbatim}

Так можно сразу подготовить для студентов вариант с решениями, а в версии для контрольной отключить отображение решений.

Мы публикуем авторские решения и критерии проверки после контрольных. Для файла с критериями мы используем окружения для указаний, описания критериев проверки и рабочих обсуждений критериев, аналогичные \texttt{sol}: \texttt{ukaz}, \texttt{crit}, \texttt{discussion}.


\end{document}

\begin{Verbatim}[frame=single]
	
\end{Verbatim}
